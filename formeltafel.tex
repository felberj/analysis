\section{Formeltafel}
\subsection{Mitternachtsformel}
\[ a x^2 + b x + c = 0 \qquad \implies \qquad x_{1,2} = \frac{-b \pm
\sqrt{b^2-4ac}}{2a} \]

\subsection{Binomialkoeffizient}
\[ \binom nk = \frac{n!}{k!\,(n-k)!} \quad \mbox{für }\ 0\leq k\leq n \]

\subsection{Kreisfunktionen}
% sin cos tan
{\footnotesize
\begin{tabular}{|l||c|c|c|c|c|c|c||c|c|}\hline
\multirow{2}{*}{$\alpha$} & $0$ & $\frac{\pi}{6}$ & $\frac{\pi}{4}$ &
$\frac{\pi}{3}$ & $\frac{\pi}{2}$ & $\frac{2\pi}{3}$ & $\pi$ &
\multirow{2}{*}{Periode} & \multirow{2}{*}{Wertebereich}\\

& $0^\circ$ & $30^\circ$ & $45^\circ$ & $60^\circ$ & $90^\circ$ & $120^\circ$ &
$180^\circ$ & &\\ \hline

$\sin$ & $0$ & $\frac{1}{2}$ & $\frac{\sqrt{2}}{2}$ &
$\frac{\sqrt{3}}{2}$ & $1$ & $\frac{\sqrt{3}}{2}$ & $0$ & $\sin(\alpha +
k\cdot$\hl{$2\pi$}$)$ & $[-1,1]$\\ \hline

$\cos$ & $1$ & $\frac{\sqrt{3}}{2}$ & $\frac{\sqrt{2}}{2}$ & $\frac{1}{2}$ & $0$
& $-\frac{1}{2}$ & $-1$ & $\cos(\alpha + k\cdot$\hl{$2\pi$}$)$ & $[-1,1]$\\
\hline


$\tan$ & $0$ & $\frac{\sqrt{3}}{3}$ & $1$ & $\sqrt{3}$ & $\pm \infty$ &
$-\sqrt{3}$ & $0$ & $\tan(\alpha + k \cdot$\hl{$\pi$}$)$ & $\R$
\\
\hline
\end{tabular}
}
\includegraphics[width=\columnwidth]{sin_cos_tan.eps}

% sinh cosh tanh
% 2 columns (don't insert a new linew between minipage definition, otherwise the layout will break!)
% In order to place the text on top/middle of column next to the image \vspace{0pt} must be inserted!
% Otherwise the text will appear on bottom! Go figure..
\begin{minipage}{0.25\textwidth}
	\vspace{0pt}
	{\footnotesize
	\begin{tabular}{|l||c|c|c|c|c||c|c|}\hline
		 & 0 & $\frac{1}{2}$ & $\frac{\sqrt{2}}{2}$ & $\frac{\sqrt{3}}{2}$ & 1 & DB & WB\\ \hline
		$\arcsin$ & 0 & $\frac{\pi}{6}$ & $\frac{\pi}{4}$ & $\frac{\pi}{3}$ & $\frac{\pi}{2}$ & 
		$[-1, 1]$ & $[-\frac{\pi}{2}, \frac{\pi}{2}]$\\ \hline 

		$\arccos$ & $\frac{\pi}{2}$ & $\frac{\pi}{3}$ & $\frac{\pi}{4}$ & $\frac{\pi}{6}$ & $0$ & 
		$[-1, 1]$ & $[0, \pi]$\\ \hline 

		$\arctan$ & $0$ & - & - & - & $\frac{\pi}{4}$ & $\R$ & $]- \frac{\pi}{2}, \frac{\pi}{2}[$\\ \hline 

		$\sinh$ & 0 & - & - & - & - & $\R$ & $\R$\\ \hline
		$\cosh$ & 1 & - & - & - & - & $\R$ & $[1, \infty[$\\ \hline
		$\tanh$ & 0 & - & - & - & - & $\R$ & $]-1, 1[$\\ \hline
	\end{tabular}
	}
\end{minipage}
\begin{minipage}{0.25\textwidth}
		\vspace{0pt}
		\hspace{1.05cm}\includegraphics[scale=0.22]{sinh_cosh_tanh_big.png}
\end{minipage}


\subsubsection{Einheitskreis}
\includegraphics[width=\columnwidth]{einheitskreis_sin_cos.eps}
$\tan \alpha = \frac{\sin \alpha}{\cos \alpha} = \frac{y}{x}$
\pagebreak

\subsection{Trigonom. Funktionen \& Additionstheoreme}
\begin{itemize}[leftmargin=*]
	\item $\sin^2(x) + \cos^2(x) = 1$
	
	% sinus equations
	\item $\sin(90^\circ \pm \alpha) = \cos(\alpha)$
	\item $\sin(180^\circ \pm \alpha) = \mp \sin(\alpha)$
	\item $\sin(\alpha \pm \beta) = \sin(\alpha)\cos(\beta) \pm	\cos(\alpha)\sin(\beta)$
	\item $\sin(2\alpha) = 2 \sin(\alpha)\cos(\alpha)$
	\item $\sin(\alpha)^2 = \frac{1}{2} (1 - \cos(2\alpha))$

	% cosinus equations
	\item $\cos(90^\circ \pm \alpha) = \mp \sin(\alpha)$
	\item $\cos(180^\circ \pm \alpha) = - \cos(\alpha)$
	\item $\cos(\alpha \pm \beta) = \cos(\alpha)\cos(\beta) \mp \sin(\alpha) \sin(\beta)$
	\item $\cos(2\alpha) = \cos^2(\alpha) - \sin^2(\alpha) = 2 \cos^2(\alpha) - 1 = 1 - 2 \sin^2(\alpha)$
	\item $\cos(\alpha)^2 = \frac{1}{2} (1 + \cos(2\alpha))$
	\item $\frac{1}{\cos^2(\alpha)} = 1 + \tan^2(\alpha)$

	% tangens equations
	\item $\tan(\alpha \pm \beta) = \frac{\tan(\alpha) \pm \tan(\beta)}{1 \mp \tan(\alpha)\tan(\beta)}$
	\item $\tan(2\alpha) = \frac{2 \tan(\alpha)}{1 - \tan^2(\alpha)}$
	\item $\tan(\alpha) = \frac{\sin(\alpha)}{\cos{(\alpha)}}$
\end{itemize}

\subsection{Hyperbelfunktionen}
\begin{itemize}[leftmargin=*]
	\item $\sinh(x) = \frac{1}{2}(e^x - e^{-x})$
	\item $\sinh(2x) = 2 \sinh(x) \cosh(x)$
	\item $\sinh(x)^2 = \frac{1}{2} (\cosh(2x) - 1)$
	\item $1 + \sinh(x)^2 = \cosh(x)^2$ $\Leftrightarrow$ $\sqrt{1 + \sinh(x)^2} = \cosh(x)$
	\item $\cosh(x)^2 - \sinh(x)^2 = 1$
	\item $\cosh(x) = \frac{1}{2}(e^x + e^{-x})$
	\item $\cosh(x)^2 = \frac{1}{2} (\cosh(2x) + 1)$
	\item $\tanh(x) = \frac{\sinh(x)}{\cosh(x)} = \frac{e^x - e^{-x}}{e^x +
	e^{-x}} = \frac{e^{2x} - 1}{e^{2x} + 1} = 1 - \frac{2}{e^{2x} + 1}$
	\item $\arcsinh(x) = \ln(x + \sqrt{x^2 + 1})$
	\item $\arcosh(x) = \ln(x + \sqrt{x^2 - 1})$
	\item $\arctanh(x) = \frac{1}{2} \ln(\frac{1+x}{1-x})$
	\item Umformung: $\tanh(x) + 1 = \frac{e^{2x} - 1}{e^{2x} + 1} + 1 = \frac{2x -
	1 + e^{2x} + 1}{e^{2x} + 1} = \frac{2e^{2x}}{e^{2x} + 1}$
\end{itemize}


\subsection{Ableitungen}
{\small
\begin{itemize}[leftmargin=*]
	\item (Summenregel) $(f + g)'(x) = f'(x) + g'(x)$
	\item (Produktregel) $(fg)'(x) = f'(x) \cdot g(x) + f(x) \cdot g'(x)$
	\item (Quotientenregel) $(\frac{f}{g})'(x) = \frac{f'(x) \cdot g(x) -
	f(x) \cdot g'(x)}{g^2(x)}$
	\item (Kettenregel) $(g \circ f)'(x) = (g(f(x)))' = g'(f(x)) \cdot f'(x)$
\end{itemize}}

\subsubsection{Ableitungs-Tafel}
\begin{itemize}[leftmargin=*]
	\item $\frac{d}{dx}\; x^n = nx^{n-1}$
	\item $\frac{d}{dx}\; \frac{1}{x^n} = -n \frac{1}{x^{n+1}}$
	
	% sqrt
	\item 
	\begin{minipage}{0.4\columnwidth}
		$\frac{d}{dx}\; \sqrt{x} = \frac{1}{2\sqrt{x}}$
	\end{minipage}
	\begin{minipage}{0.55\columnwidth}
		$\frac{d}{dx}\; \sqrt[n]{x} = \frac{1}{n\sqrt[n]{x^{n-1}}}$
	\end{minipage}
	
	% e^x
	\item
	\begin{minipage}{0.4\columnwidth}
		$\frac{d}{dx}\; e^x = e^x$
	\end{minipage}
	\begin{minipage}{0.55\columnwidth}
		$\frac{d}{dx}\; e^{\alpha x + \beta} = \alpha e^{\alpha x + \beta}$
	\end{minipage}
	
	\item $\frac{d}{dx}\; e^{x^\alpha} = \alpha x^{\alpha - 1} e^{x^\alpha}$


	\item $\frac{d}{dx}\; \ln(x) = \frac{1}{x}$
	\item $\frac{d}{dx}\; \alpha^x = \alpha^x \ln(\alpha)$
	\item $\frac{d}{dx}\; x^x = x^x (1 + \ln(x))$
	\item $\frac{d}{dx}\; x^{x^\alpha} = x^{x^\alpha + \alpha - 1} (\alpha \log(x) + 1)$ 
	\vspace{0.1cm}\hrule

	% sin cos tan
	\item 
	\begin{minipage}{0.4\columnwidth}
		$\frac{d}{dx}\; \sin(x) = \cos(x)$	
	\end{minipage}
	\begin{minipage}{0.55\columnwidth}
		$\frac{d}{dx}\; \sin(\alpha x + \beta) = \alpha \cos(\alpha x + \beta)$	
	\end{minipage}
	
	
	\item 
	\begin{minipage}{0.4\columnwidth}
		$\frac{d}{dx}\; \cos(x) = -\sin(x)$
	\end{minipage}
	\begin{minipage}{0.55\columnwidth}
		$\frac{d}{dx}\; \cos(\alpha x + \beta) = -\alpha \sin(\alpha x + \beta)$
	\end{minipage}

	\item 
	\begin{minipage}{0.4\columnwidth}
		$\frac{d}{dx}\; \tan(x) = \frac{1}{cos^2(x)}$	
	\end{minipage}
	\begin{minipage}{0.55\columnwidth}
		$\frac{d}{dx}\; \tan(\alpha x + \beta) = \alpha \frac{1}{\cos^2(\alpha x + \beta)}$	
	\end{minipage}
	
	\item $\frac{d}{dx}\; \tan(x) = 1 + \tan(x)^2$

	\item 
	\begin{minipage}{0.4\columnwidth}
		$\frac{d}{dx}\; \arcsin(x) = \frac{1}{\sqrt{1-x^2}}$	
	\end{minipage}
	\begin{minipage}{0.55\columnwidth}
		$\frac{d}{dx}\; \arcsin(\alpha x + \beta) = \frac{\alpha}{\sqrt{1-(\alpha x + \beta)^2}}$
	\end{minipage}
		
	\item 
	\begin{minipage}{0.4\columnwidth}
		$\frac{d}{dx}\; \arccos(x) = -\frac{1}{\sqrt{1-x^2}}$ 
	\end{minipage}
	\begin{minipage}{0.55\columnwidth}
		$\frac{d}{dx}\; \arccos(\alpha x + \beta) = -\frac{\alpha}{\sqrt{1 - (\alpha x + \beta)^2}}$
	\end{minipage}
	
	\item 
	\begin{minipage}{0.4\columnwidth}
		$\frac{d}{dx}\; \arctan(x) = \frac{1}{x^2+1}$
	\end{minipage}
	\begin{minipage}{0.55\columnwidth}
		$\frac{d}{dx}\; \arctan(\alpha x + \beta) = \frac{\alpha}{(\alpha x + \beta)^2 + 1}$
	\end{minipage}
	\vspace{-0.2cm}
	\hrule
	\item 
	{\small
	\begin{minipage}{0.4\columnwidth}
		$\frac{d}{dx}\; \sinh(x) = \cosh(x)$
	\end{minipage}
	\begin{minipage}{0.55\columnwidth}
		$\frac{d}{dx}\; \sinh(\alpha x + \beta) = \alpha \cosh(\alpha x + \beta)$
	\end{minipage}}

	\item
	{\small
	\begin{minipage}{0.4\columnwidth}
		$\frac{d}{dx}\; \cosh(x) = \sinh(x)$
	\end{minipage}
	\begin{minipage}{0.55\columnwidth}
		$\frac{d}{dx}\; \cosh(\alpha x + \beta) = \alpha \sinh(\alpha x + \beta)$
	\end{minipage}}
	
	\item 
	{\small
	\begin{minipage}{0.4\columnwidth}
		$\frac{d}{dx}\; \tanh(x) = \frac{1}{\cosh^2(x)}$
	\end{minipage}
	\begin{minipage}{0.55\columnwidth}
		$\frac{d}{dx}\; \tanh(\alpha x + \beta) = \alpha \frac{1}{\cosh^2(\alpha x + \beta)}$
	\end{minipage}}

	\item 
	{\small
	\begin{minipage}{0.4\columnwidth}
		$\frac{d}{dx}\; \arcsinh(x) = \frac{1}{\sqrt{x^2 + 1}}$
	\end{minipage}
	\begin{minipage}{0.55\columnwidth}
		$\frac{d}{dx}\; \arcsinh(\alpha x + \beta) = \frac{\alpha}{\sqrt{(\alpha x + \beta)^2 + 1}}$
	\end{minipage}}
	
	\item
	{\footnotesize
	\begin{minipage}{0.4\columnwidth}
		$\frac{d}{dx}\; \arcosh(x) = \frac{1}{\sqrt{x-1} \sqrt{x + 1}}$
	\end{minipage}
	\begin{minipage}{0.55\columnwidth}
		$\frac{d}{dx}\; \arcosh(\alpha x + \beta) = \frac{\alpha}{\sqrt{\alpha x + \beta - 1} \sqrt{\alpha x + \beta + 1}}$
	\end{minipage}}

	\item 
	{\small
	\begin{minipage}{0.4\columnwidth}
		$\frac{d}{dx}\; \arctanh(x) = \frac{1}{1-x^2}$
	\end{minipage}
	\begin{minipage}{0.55\columnwidth}
		$\frac{d}{dx}\; \arctanh(\alpha x + \beta) = \frac{\alpha}{1 - (\alpha x + \beta)^2}$
	\end{minipage}}
\end{itemize}

\subsection{Integrale}
\subsubsection{Integralregeln}
\begin{itemize}[leftmargin=*]
	\item $\int u'\cdot v \, dx = uv - \int u \cdot v' \, dx$
	\item $\int \frac{f'(x)}{f(x)} \, dx = \ln|f(x)|$
	\item $\int f(x)f'(x) \, dx = \frac{1}{2}f(x)^2$
	\item $|\int f(x)| \leq \int |f(x)|$ (wenn f, Riemann-Integrable ist)
\end{itemize}

\subsubsection{typische Integrale}
\begin{itemize}[leftmargin=*]
	% typ: x^n
	\item $\int x^n \, dx = \frac{x^{n+1}}{n+1}$ \hspace{0.3cm} für $n \neq -1$
  	\item $\int(ax + b)^n \,dx = \frac{(ax + b)^{n+1}}{(n + 1)a}$ \hspace{0.3cm} für $n \neq -1$
	\item $\int x(ax+b)^n \,dx = \frac{(ax + b)^{n+2}}{(n+2)a^2} - \frac{b(ax+b)^{n+1}}{(n+1)a^2}$

  	% typ: 1/x resulting in ln
  	\item $\int \frac{1}{x} \,dx = \ln |x|$
	\item $\int \frac{1}{x^2} \,dx = -\frac{1}{x}$
  	\item $\int \frac{1}{a+x} \,dx = \ln |a+x|$
  	\item $\int \frac{1}{(a+x)^2} \,dx = - \frac{1}{a+x}$
	\item $\int \frac{1}{ax+b} \,dx = \frac{1}{a} \ln |ax+b|$
	\item $\int \frac{x}{1 + x^2} \, dx = \frac{1}{2} \ln |1 + x^2|$

	% typ: 1/x resulting in arctan
	\item $\int \frac{1}{1 + x^2} \, dx = \arctan(x)$
	\item $\int \frac{1}{a^2 + x^2} \,dx = \frac{1}{a} \arctan(\frac{x}{a})$
	\item $\int \frac{1}{a^2 - x^2} \,dx = \frac{1}{a} \arctanh(\frac{x}{a})$
	\item $\int \frac{1}{1 + (a + x)^2} \, dx = \arctan(a + x)$
	
	
	% typ: ln
  	\item $\int \ln(x) \,dx = x(\ln(x) - 1)$
  	\item $\int \ln(ax + b) \,dx = \frac{(a x+b) \ln (a x+b)-a x}{a}$
	
	% typ: sqrt
	\item $\int \sqrt{x} \,dx = \frac{2}{3}\sqrt{x^3}$
	\item $\int \frac{1}{\sqrt{x}} \,dx = 2 \sqrt{x}$	
	%mühsamer kerl der teilweise in prüfungen verwendet wird. Kann man über subsitution von x mit sin(u) lösen.
	\item $\int \sqrt{1-x^2} \,dx = \frac{1}{2}\left( x\sqrt{1-x^2}+ \arcsin(x) \right)$
	\item $\int \frac{1}{\sqrt{1 - x^2}} \, dx = \arcsin(x)$
	\item $\int \frac{1}{\sqrt{1 + x^2}} \, dx = \arcsinh(x)$

	\item $\int a^{xb + c} \,dx = \frac{a^{bx + c}}{b \log(a)}$
	\item $\int \frac{ax + b}{px + q} \,dx = \frac{ax}{p} + \frac{bp - aq}{p^2} \ln |pq+q|$
\end{itemize}

\subsubsection{trionometrische Funktionen}
\begin{itemize}[leftmargin=*]
	% sin
	\item
	\begin{minipage}{0.5\columnwidth}
		$\int \sin(x) \, dx = - \cos(x)$
	\end{minipage}
	\begin{minipage}{0.45\columnwidth}
		$\int \sin(ax) \,dx = -\frac{1}{a}\cos(ax)$
	\end{minipage}

	\item
	\begin{minipage}{0.5\columnwidth}
		 $\int \sin(x)^2 \, dx = \frac{x}{2} - \frac{\sin(x) \cos(x)}{2}$
	\end{minipage}
	\begin{minipage}{0.45\columnwidth}
		$\int \sin(ax)^2 \,dx = \frac{x}{2} - \frac{sin(2ax)}{4a}$
	\end{minipage}
		  
	\item $\int x \sin(ax) \,dx = \frac{\sin(ax)}{a^2} - \frac{x \cos(ax)}{a}$
	\item $\int \frac{1}{\sin^2 x} \,dx = -\cot x$
   	\item $\int \sin(ax) \cos(ax) \,dx = -\frac{\cos^2(ax)}{2a}$

   	% cos
   	\item 
	\begin{minipage}{0.5\columnwidth}
		$\int \cos(x) \, dx = \sin(x)$
	\end{minipage}
	\begin{minipage}{0.45\columnwidth}
		$\int \cos(ax) \,dx = \frac{1}{a}\sin(ax)$
	\end{minipage}

	\item
	\begin{minipage}{0.5\columnwidth}
		$\int \cos(x)^2 \, dx = \frac{x}{2} + \frac{\sin(x) \cos(x)}{2}$
	\end{minipage}
	\begin{minipage}{0.45\columnwidth}
		$\int \cos^2(ax) \,dx = \frac{x}{2} + \frac{\sin(2ax)}{4a}$
	\end{minipage}
	\item $\int x \cos(ax) \,dx = \frac{\cos(ax)}{a^2} + \frac{x \sin(ax)}{a}$
	\item $\int \frac{1}{\cos^2(x)} \,dx = \tan x$
	
	% rest
	\item $\int \tan(ax) \,dx = - \frac{1}{a} \ln | \cos(ax) |$
	\item $\int \arcsin(x) \,dx = x \arcsin(x) + \sqrt{1 - x^2}$
	\item $\int \arccos(x) \,dx = x \arccos(x) - \sqrt(1-x^2)$
	\item $\int \arctan(x) \,dx = x \arctan(x) - \frac{1}{2} \ln(1+x^2)$
\end{itemize}

\subsubsection{Hyperbelfunktionen}
\begin{itemize}[leftmargin=*]
	\item 
	\begin{minipage}{0.43\columnwidth}
		$\int \sinh(x) \,dx = \cosh(x)$
	\end{minipage}
	\begin{minipage}{0.52\columnwidth}
		$\int \sinh(ax + b) \,dx = \frac{\cosh(ax + b)}{a}$
	\end{minipage}

	\item
	\begin{minipage}{0.43\columnwidth}
		$\int \cosh(x) \,dx = \sinh(x)$
	\end{minipage}
	\begin{minipage}{0.52\columnwidth}
		$\int \cosh(ax + b) \,dx = \frac{\sinh(ax + b)}{a}$
	\end{minipage}
	
	\item
	\begin{minipage}{0.43\columnwidth}
		$\int \tan(x) \,dx = \log(\cosh(x))$
	\end{minipage}
	\begin{minipage}{0.52\columnwidth}
		$\int \tan(ax + b) \,dx = \frac{\log(\cosh(ax+b))}{a}$
	\end{minipage}
\end{itemize}

\subsubsection{Exponentialfunktion}
\begin{itemize}[leftmargin=*]
  	\item $\int e^{ax} \,dx = \frac{1}{a} e^{ax}$ 
	\item $\int x e^{ax} \,dx = e^{ax} \cdot \left ( \frac{ax - 1}{a^2} \right )$
	\item $\int x \ln(x) \,dx = \frac{1}{2} x^2 (\ln(x) - \frac{1}{2})$
	\item $\int_{-\infty}^\infty e^{-\frac{1}{a}x^2} \,dx = \sqrt{a \pi}$
\end{itemize}
\pagebreak
\subsection{Reihenentwicklung}
\begin{itemize}[leftmargin=*]
	\item $e^x = \sum_{n=0}^\infty \frac{x^n}{n!} = 1 + \frac{x}{1!} + \frac{x^2}{2!} + \frac{x^3}{3!} + 
				 \frac{x^4}{4!} + \cdots$
	\item $\sin x = \sum_{n=0}^\infty (-1)^n \frac{x^{2n + 1}}{(2n + 1)!} = x - \frac{x^3}{3!} + \frac{x^5}{5!} + \cdots$
	\item $\cos x = \sum_{n=0}^\infty (-1)^n \frac{x^{2n}}{(2n)!} = 1 - \frac{x^2}{2!} + \frac{x^4}{4!} - \cdots + \cdots$
	\item $\sinh x = \sum_{n=0}^\infty \frac{x^{2n+1}}{(2n + 1)!}$
	\item $\cosh x = \sum_{n=0}^\infty \frac{x^{2n}}{(2n)!}$
	\item $\ln x = \sum_{n=0}^\infty \frac{2}{2n + 1} \cdot \left(\frac{x-1}{x+1} \right)^{2n}$
\end{itemize}

\subsection{Grenzwerte}
\begin{itemize}[leftmargin=*]
	\item \textbf{Bernoullische Ungleich.}: $x \geq -1, n \in \N: \; (1+x)^n \geq 1+nx$
	\item \textbf{Vergleich von Folgen}: weiter rechts stehende Folgen streben
	schneller gegen $\infty$:
	\[
		1, \quad \ln n, \quad n^\alpha (\alpha > 0), \quad q^n (q > 1), \quad n!, \quad n^n \Rightarrow
		\lim_{x \to \infty} \frac{\ln n}{n^\alpha} = 0
	\]
\end{itemize}
$\lim_{n \to \infty}$
\begin{itemize}[leftmargin=*]
	\item $\lim_{n \to \infty} \sqrt[n]{a} \rightarrow 1$
	\item $\lim_{n \to \infty} \sqrt[n]{n} \rightarrow 1$
	\item $\lim_{n \to \infty} \sqrt[n]{n!} \rightarrow \infty$
	\item $\lim_{n \to \infty} \frac{n}{\sqrt[n]{n!}} \rightarrow e$
	\item $\lim_{n \to \infty} \frac{1}{n} \sqrt[n]{n!} \rightarrow \frac{1}{e}$
	\item $\lim_{n \to \infty} \left ( \frac{n+1}{n} \right )^n \rightarrow e$
	\item $\lim_{n \to \infty} \left ( 1 + \frac{1}{n} \right )^n \rightarrow e$
	\item $\lim_{n \to \infty} \left ( 1 - \frac{1}{n} \right )^n \rightarrow \frac{1}{e}$
	\item $\lim_{n \to \infty} \left ( \frac{n}{1 + n} \right )^n \rightarrow \frac{1}{e}$
	\item $\lim_{n \to \infty} \left ( 1 + \frac{x}{n} \right )^n \rightarrow e^x$
	\item $\lim_{n \to \infty} \left ( 1 - \frac{x}{n} \right )^n \rightarrow \frac{1}{e^x}$
	\item $\lim_{n \to \infty} {a \choose n} \rightarrow 0, \; a > -1$
	\item $\lim_{n \to \infty} \frac{a^n}{n!} \rightarrow 0$
	\item $\lim_{n \to \infty} \frac{n^n}{n!} \rightarrow \infty$
	\item $\lim_{n \to \infty} \frac{a^n}{n^k} \rightarrow \infty, a > 1, k$ fest
	\item $\lim_{n \to \infty} a^n n^k \rightarrow 0, |a| < 1, k$ fest
	\item $\lim_{n \to \infty} n(\sqrt[n]{a} - 1) \rightarrow \ln a, a > 0$
	\item $\lim_{n \to \infty} \left( 1+\frac{x}{n} \right)^n = e^x \quad$
	\item $\lim_{n \to \infty} \sqrt[n]{n} = 1$
	\item $\lim_{n \to \infty} n^p q^n = 0 \qquad p \in \N \text{ und } 0 < q < 1$
	\item $\lim_{x \to \infty} \sqrt{x^2-x}-x = \frac{1}{2}$ 
	%\newline
	%{\small (Lösungsansatz mit Taylorreihe
	%($\sqrt{1-x} = 1 + \frac{x}{2}+O(x^2)$): $\sqrt{x^2-x}-x = x(\sqrt{1-\frac{1}{x}}-1) =
	%x((1+\frac{1}{2x}+O(\frac{1}{x^2}))-1) = \frac{1}{2}+O(\frac{1}{x}) \underset{n \to \infty}{\longrightarrow} \frac{1}{2}$ )}
\end{itemize}
$\lim_{x \to 0}$
\begin{itemize}[leftmargin=*]
	\item $\lim_{x \to 0} \frac{a^x - 1}{x} = \ln a$
	\item $\lim_{x \to 0} \frac{\sin x}{x} = 1$
	\item $\lim_{x \to 0} \left| \frac{\cos x}{x} \right| = +\infty$
	\item $\lim_{x \to 0} \frac{1 - \cos x}{x} = 0$
	\item $\lim_{x \to 0} \frac{1 - \cos x}{x^2} = \frac{1}{2}$
	\item $\lim_{x \to 0} \frac{\tan x}{x} = 1$
	\item $\lim_{x \to 0} \frac{\log_a (1 + x)}{x} = \frac{1}{\ln a}$
	\item $\lim_{x \to 0} x^a \ln x = 0, \; a  > 0$
\end{itemize}

\subsection{Reihen}
\begin{itemize}[leftmargin=*]
	\item
	\begin{minipage}{0.35\columnwidth}
		$\sum_{n=1}^\infty \frac{1}{n}$
	\end{minipage}
	\begin{minipage}{0.60\columnwidth}
		divergiert (``harmonische Reihe'')
	\end{minipage}

	\item $\sum_{n=1}^\infty \frac{(-1)^n}{n} = \ln \frac{1}{2}$
	\item 
	\begin{minipage}{0.35\columnwidth}
		$\sum_{n=1}^\infty \frac{1}{n^\alpha}$
	\end{minipage}
	\begin{minipage}{0.60\columnwidth}
		konver. für $\alpha > 1$, divergiert für $\alpha \leq 1$
	\end{minipage}
	 
	\item 
	\begin{minipage}{0.35\columnwidth}
		$\sum_{n=0}^\infty q^n = \frac{1}{1-q}$
	\end{minipage}
	\begin{minipage}{0.60\columnwidth}
		für $|q| < 1$ (``geometrische Reihe'')
	\end{minipage}
	
	\item
	\begin{minipage}{0.35\columnwidth}
		$\sum_{n=0}^\infty (-1)^n q^n = \frac{1}{1-q}$
	\end{minipage}
	\begin{minipage}{0.60\columnwidth}
		für $|q| < 1$ (``geometrische Reihe'')
	\end{minipage}

	\item $\sum_{n=1}^\infty \frac{1}{n^2} = \frac{\pi^2}{6}$
	\newline
	\item $\sum_{n=0}^m q^n = \frac{1-q^{m+1}}{1-q}$ 
	\item $\sum_{n=1}^m n = \frac{m(m+1)}{2}$
	\item  $\sum_{n=0}^m n^2 = \frac{1}{6}m(m+1)(2m+1)$
	\item  $\sum_{n=0}^m n^3 = \frac{1}{4}m^2(m+1)^2$
\end{itemize}

\subsection{Linienintegral}
\begin{itemize}[leftmargin=*]
	\item 2. Art: $\int_\gamma \vec{f}(\vec{x}) d\vec{x} := \int_a^b \left<
	\vec{f}(\gamma(t)), \gamma(t)' \right>\; dt$
	\item 1. Art: $\int_\gamma f ds := \int_a^b f(\gamma(t)) \|\gamma(t)'\|_2\; dt$
\end{itemize}

\subsection{Kreuzprodukt}
{\footnotesize
\[
\vec{a} \times \vec{b} = \left ( \begin{array}{c} a_1 \\ a_2 \\ a_3 \end{array}
\right ) \times
\left ( \begin{array}{c} b_1 \\ b_2 \\ b_3 \end{array}
\right ) =
\left ( \begin{array}{c} a_2b_3 - a_3b_2 \\ a_3b_1 - a_1b_3 \\ a_1b_2 - a_2b_1
\end{array} \right )
\]
}

\begin{multicols}{2}
\subsection{Exponent}
\begin{itemize}[leftmargin=*]
  \item $a^n a^m = a^{n + m}$
  \item $\frac{a^n}{a^m} = a^{n - m}$
  \item $(a^n)^m = a^{nm}$
  \item $(ab)^n = a^n b^n$
  \item $\left( \frac{a}{b} \right)^n = \frac{a^n}{b^n}$
  \item $a^{-n} = \frac{1}{a^n}$
  \item $\left( \frac{a}{b} \right)^{-n} = \left( \frac{b}{a} \right)^n$
  \item $a^\frac{n}{m} = (a^\frac{1}{m})^n = (a^n)^\frac{1}{m}$
  \item $a^{n^m} = a^{(n^m)}$
\end{itemize}
\columnbreak

\subsection{Wurzel}
\begin{itemize}[leftmargin=*]
  \item $\sqrt[n]{a} = a^\frac{1}{n}$
  \item $\sqrt[n]{ab} = \sqrt[n]{a} \sqrt[n]{b}$
  \item $\sqrt[m]{\sqrt[n]{a}} = \sqrt[nm]{a}$
  \item $\sqrt[n]{\frac{a}{b}} = \frac{\sqrt[n]{a}}{\sqrt[n]{b}}$
\end{itemize}

\end{multicols}

\subsection{Ungleichungen}
\begin{itemize}[leftmargin=*]
  \item $a < b \Rightarrow a + c < b + c$ und $a - c < b - c$
  \item $a < b$ und $c > 0 \Rightarrow \frac{a}{c} < \frac{b}{c}$
  \item $a < b$ und $c < 0 \Rightarrow \frac{a}{c} > \frac{b}{c}$ 
  \item Dreiecksungleichung für reelle Zahlen: $|a+b| \le |a|{+}|b|$ %Quelle Wikipedia: http://de.wikipedia.org/wiki/Dreiecksungleichung#Dreiecksungleichung_f.C3.BCr_reelle_Zahlen
  \item Cauchy-Schwarz Ungleichung: $|x \cdot y| \leq \|x\| \cdot \|y\|, \; x,y \in \R^n$
\end{itemize}

\subsection{Logarithmen}
\begin{itemize}[leftmargin=*]
  \item $y = \log_a x \Leftrightarrow x = a^y$
  \item $\log_a 1 = 0$
  \item $\log_a a^x = x$
  \item $a^{\log_a x} = x $
  \item $\log_a x \cdot y = \log_a x + \log_a y$
  \item $\log_a \frac{x}{y} = \log_a x - \log_a y$
  \item $\log_a \frac{1}{x} = - \log_a x$
  \item $\log_a x^r = r \log_a x$
  \item $\log_a x = \frac{\log_b x}{\log_b a}$
  \item $\log_a x = \frac{\ln x}{\ln a}$
  \item $\log_a (x+y) = \log_a x + log_a (1 + \frac{y}{x})$
  \item $\log_a (x-y) = \log_a x + \log_a (1- \frac{y}{x})$
\end{itemize}

\subsection{Exponentialfunktion}
\begin{itemize}[leftmargin=*]
  \item $e^{-\inf} = 0$
  \item $e^0 = 1$
  \item $e^1 = e =  2.718281828$
  \item $e^{\inf} = \inf$
  \item $e^{a+bi} = e^a(\cos(b) + i \sin(b))$ (Euler Identität)
  \item $e^{b \ln(a)} = a^b$
  \item $ e^{-\ln(b)} = \frac{1}{b}$ 
\end{itemize}

\subsection{Komplexe Zahlen}
\begin{itemize}[leftmargin=*]
	\item $z \in \C: z = a + b\cdot i$
	\item $\bar{z} = a - b\cdot i$
	\item $|z|^2 = z \cdot \bar{z} = (a + b\cdot i) \cdot (a - b\cdot i) = a^2 + b^2$
	\item $i^2 = -1$
	\item $(a + bi) + (c + di) = (a + c) + (b + d)i$
	\item $(a + bi) \cdot (c + di) = (ac - bd) + (ad + bc)i$
	\item $\frac{a + bi}{c + di} = \frac{ac + bd}{c^2 + d^2} + \frac{bc - ad}{c^2 + d^2}\cdot i$
\end{itemize}
\pagebreak
\subsection{Geometrische Körper}
\subsubsection{Ellipsoid}
Hat die Form eines Rugbyballs. In kartesischen Koordinaten definert durch
$\frac{x^2}{a^2} + \frac{y^2}{b^2} + \frac{z^2}{c^2} - 1 = 0$.

\subsection{Geometrie in 3D}
%evtl. nicht nötig. Aber in alten prüfungen oft gefragt.
\subsubsection*{Masse von speziellen Gebieten}
	\begin{multicols}{2}
	\renewcommand\arraystretch{1.4}
	\begin{tabular}{l|l}
		Zylinder   &   $ V = \pi r^2 h $ \\
		Pyramide   &   $ V = \frac{1}{3} G h $ \\
		Ellipsoid   &   $ V = \frac{4 \pi}{3} a b c $ \\
		Kegel   &   $ V = \frac{\pi}{3} r^2 h $ \\
		Kegelstumpf   &   $ V = \frac{\pi h}{3} (r_1^2+r_2^2 + r_1 r_2) $ 
	\end{tabular}
	
	\columnbreak
	\begin{tabular}{l|l}
		Torus   &   $ V = 2 \pi^2 R r^2 $ \\
		   &   $ S = 4 \pi^2 R r $ \\
		Kugel   &   $ V = \frac{4 \pi}{3} r^3 $ \\
		   &   $ S = 4 \pi r^2 $ 
	\end{tabular}
	\end{multicols}

\subsubsection*{Rotationskörper (Volumen / Mantelfläche)}
Rotation um die x Achse $V =\pi \int_a^b f(x)^2 dx.$\\
Rotation um die x Achse $M = 2\pi \int_a^b f(x) \, \sqrt{1 + f'(x)^2} dx$\\
{\small \textbf{Achtung:} Für ganze Fläche muss Deckel dazu berechnet werden!}
% Rotation um die y Achse $V=2\pi \int_a^b x \cdot f(x) dx$ => falsch!

\subsection{Kosinussatz}
\begin{minipage}{0.35\columnwidth}
	\includegraphics[width=\columnwidth]{kosinussatz.jpg}
\end{minipage}
\begin{minipage}{0.60\columnwidth}
	\begin{eqnarray*}
		a^2 = b^2 + c^2 - 2bc \, \cdot \, \cos(\alpha)\\
		b^2 = a^2 + c^2 - 2ac \, \cdot \, \cos(\beta)\\
		c^2 = a^2 + b^2 - 2ab \, \cdot \, \cos(\gamma)
	\end{eqnarray*}
\end{minipage}

\subsection{Ausklammern}
\begin{itemize}[leftmargin=*]
	\item $x^n - y^n = (x-y) (x^{n-1} + x^{n-2}y + x^{n-3}y^2 + \ldots + xy^{n-2}
	+ y^{n-1})$
	\item $x^n - 1 = (x-1)(x^{n-1} + x^{n-2} + \ldots + x + 1)$
\end{itemize}

\subsection{Aus Serien}
\begin{itemize}[leftmargin=*]
	\item Ableitung von $x^x$ kann man mit Ansatz $x = e^{\log(x)}$ berechnen.
	Also:$ \frac{d}{dx}  e^{\log(x^x)} = \frac{d}{dx} e^{x \log(x)} = x^x (1 + \log(x))$
	\item Euler Identität (komplexe Zahlen): $e^{ix} = \cos(x) + i \sin(x)$
\end{itemize}

\subsection{Polynomdivision}
Zu jedem Zeitpunkt gilt: Zähler : Nenner = Ergebnis, wobei der Zähler und das Ergebnis sich nach einem Schritt jeweils ändern.
\begin{enumerate}[leftmargin=*]
	\item Prüfe ob Grad des Zählers $\geq$ Grad des Nenners ist.
	\begin{enumerate}
		\item Falls Ja:
		\begin{enumerate}
			\item Dividiere höchsten Grad vom Zähler durch höchsten Grad vom Nenner und addiere das Resultat zum Ergebnis.
			\item Multipliziere den neu hinzugefügten Summanden mit dem Nenner und notiere dieses Zwischenresultat. Danach berechne Zähler - Zwischenresultat und betrachte das als neuen Zähler. Starte wieder bei 1.
		\end{enumerate}
		\item Falls Nein:\\
		Wir sind fertig. Wenn noch ein Rest übrig bleibt, muss dies dem Ergebnis noch hinzugefügt werden. Also Rest / Nenner noch dem Erebnis addieren werden. 
	\end{enumerate}
\end{enumerate}
\textbf{Beispiel} \\
\includegraphics[scale=0.45]{polynomdivision_rechnung.png}
\includegraphics[scale=0.45]{polynomdivision_resultat.png}
